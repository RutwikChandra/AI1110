\documentclass[journal,12pt,onecolumn]{article}
\usepackage{amssymb}
\usepackage{geometry}
% Set the margins
\geometry{
  left=4cm,
  right=4cm,
  top=2cm,
  bottom=4cm,
}
\title{AI1110 Assignment-1}
\author{Rutwik Chandra Bendi (CS22BTECH11011)}

\begin{document}
\maketitle
\begin{abstract}
This document provides the solution to question 14 in Chapter 13 of the 12th grade NCERT textbook, Exercise 13.5.
\end{abstract}\par
\textbf{Question : }\par
\vspace{0.2cm} In a box containing 100 bulbs, 10 are defective. The probability that out of a sample of 5 bulbs, none is defective is\par 
(A) $10^{-1}$  (B) $\left(\frac{1}{2}\right)^5$  (C) $\left(\frac{9}{10}\right)^5$  (D) $\frac{9}{10}$ \par
\vspace{0.2cm}\textbf{Solution : }\par
\vspace{0.2cm} Let \textbf{X} be a random variable that represents the number of defective bulbs.\par
\vspace{0.2cm} This experiment of picking bulbs follows the binomial distribution.\par
\vspace{0.2cm} So, \quad \textbf{P(X=k)} = ${n\choose k}p^k (1-p)^{n-k}$\par
\vspace{0.2cm} where,\par
\vspace{0.2cm} \textbf{n} = sample size = 5\par
\textbf{k} = number of defective bulbs in the sample = 0 \par
\textbf{p} = probability of selecting a defective bulb, which is $\frac{10}{100}$ = $\frac{1}{10}$ (since there are 10 defective bulbs out of 100)\par
\vspace{0.2cm} We need to find the probability that no bulb id defective i.e \textbf{P(X=0).}\par
\vspace{0.5cm} $P(X=0) = {5\choose 0}(\frac{1}{10})^0(1-\frac{1}{10})^{5-0}$\par
\vspace{0.5cm} \hspace{1.6cm} $= 1\times1\times(\frac{9}{10})^5$\par
\vspace{0.5cm} \hspace{1.6cm} $=(\frac{9}{10})^5$\par
\vspace{0.5cm}  $\therefore$ \textbf{option(C)} is the correct answer.

\end{document}

